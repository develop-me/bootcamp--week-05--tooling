\section{Workflow considerations}

There are lots of things to consider when designing a workflow for a project.
\\

\textbf{Who?}

\begin{itemize}
    \item Who’s going to work on what? Task delegation/ownership
    \item Can the project work be chopped up into chunks/tasks?
    \item Is there a clear delegation based on team roles? Front-end, back-end, tech lead
\end{itemize}

\textbf{Timescales}

\begin{itemize}
    \item What is the project timeline?
    \item What are the milestones?
	\item Are certain tasks dependant on others?
	\item When are things going live?
\end{itemize}

\textbf{How?}

\begin{itemize}
    \item Is that work going to be reviewed? How?
	\item How is it going to be tested?
	\item How is it going merged and deployed?
	\item Are we releasing in phases?
	\item Do we quickly release new code to address bugs in the live system?
\end{itemize}

These considerations help to discern which workflow is best for your project.


\subsection{Centralised workflow}

Everyone works on master and commits all changes directly to master. This is a tricky way to work in teams.
\\

We tried an exercise doing it, I reckon most of you absolutely hated it. 
\\

It can work if you are the only person working on a project. 


\subsection{Featured branch workflow}

All feature development takes place in a dedicated branch instead of the master branch, with branches merged back into master when ready.
\\

Adding feature branches to your development process is an easy way to encourage collaboration and streamline communication. It gives other developers or lead developer the opportunity to sign off on a feature before it gets integrated into the official project.
\\

It's easy for multiple developers to work on a particular feature without disturbing the main codebase.
\\

\texttt{master} will never contain broken code, good for continuous integration environments.


\subsection{Gitflow workflow}

This, or a variation of this, is quite common in larger projects. Here's a great article about it: \href{https://nvie.com/posts/a-successful-git-branching-model/}{A successful Git branching model}.
\\

All development should take place in a branch instead of the master branch.
\\

New features should reside in their own branch and instead of branching off master, feature branches use another branch called development as their parent branch.
\\

When a feature is complete, it gets merged back into development and should never interact directly with master.
\\

Instead features end up going live because development merges to master.

\img{12cm}{resources/gitflow.jpg}{1em}{An illustration of the key gitflow principles}





\subsection{Forking workflow}

Instead of using a single server-side repository to act as the “central” codebase, it gives every developer a server-side repository.
\\

Contributions can be integrated without the need for everybody to push to a single central repository.
\\

Developers push to their own server-side repositories with only the project maintainer can push to the official repository.
\\

Allows the maintainer to accept commits from any developer without giving them write access to the official codebase.
\\

Distributed workflow that provides a flexible way for large, organic teams (including untrusted third-parties) to collaborate securely.
\\

This also makes it an ideal workflow for open source projects.



\section{Additional resources}

\begin{itemize}[leftmargin=*]
    \item \href{https://www.atlassian.com/git/tutorials/comparing-workflows}{Atlassian comparing workflows}
	\item \href{https://nvie.com/posts/a-successful-git-branching-model/}{A successful Git branching model}
\end{itemize}

