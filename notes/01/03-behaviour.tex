\section{Git Behaviour}

\subsection{(most) operations are local}

Things that only affect your computer:

\begin{itemize}
	\item adding a file
	\item deleting a file
	\item adding a commit (snapshot)
    \item rolling back to early snapshot (checking out an earlier version of your files)
    \item creating a branch
    \item creating a new snapshot
    \item merging branches
\end{itemize}

The good thing about this is it reduces requests over network, increases speed+efficiency and allows you to work offline.

\subsection{Git has integrity}

Git has a clever way of knowing if files have changed in any way, or if the files match an earlier snapshot.

\begin{itemize}
	\item Everything is check-summed before storage
	\item Snapshots are referred to by that checksum
	\item Impossible to change the contents of any file or directory without Git knowing about it
	\item Checksumming is SHA-1 hash, producing 40-character string e.g.:
24b9da6552252987aa493b52f8696cd6d3b00373
\end{itemize}

\subsection{Git only adds data}

...mostly.

It is hard to get the system to do anything that is not undoable or to make it erase data in any way.

Actions in Git (nearly) only add data to the Git database

Only way you can lose or mess up changes is if you haven’t pushed your work yet, or use some more niche commands and arguments like \texttt{--force} (more on this later).

It is therefore a great safety net for trying things out, and rolling back if needed.

After you commit a snapshot it is very difficult to lose work, especially if you regularly push to a remote repository.
