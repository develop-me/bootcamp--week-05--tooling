\section{Basic concepts}

Git won't help you manage your code unless you initialise it and tell it what's going on. It cannot organise your work for you like a magic fairy - you will always have to think about how to use its power to organise your code in a sensible way. 
\\

Getting good at this comes with experience and fluffing it up a few times.


\subsection{Repositories and commits}

When you begin work on a coding project you typically create a directory to house all your project files. Or you might use an existing one if someone else has already started work.

\begin{infobox}{A repository}
	is a directory that Git has been told to monitor. It will watch all the files and subfolders within that directory. It's usually referred to as a \textbf{repo}.
\end{infobox}

The basic building blocks of a git repo are called \textbf{commits}.

\begin{infobox}{A commit}
	A snapshot or version of your code which you intentionally take at a moment in time. 
\end{infobox}
	
Commits are named by the developer and a hash is automatically generated which looks like this \textbf{166348d5901829380ab5b44ffa09fea3a595f64d}.

By looking at the history of commits in a repository, another developer can look at the code and figure out the story of how it's gotten to where it is today.

\img{12cm}{resources/commit-history.jpg}{1em}{A commit history can tell the story of a project's code}


\subsection{File states}

In Git's world, files can have different states. It is important to know what state a file is in as it will affect what Git does with it. 
\\

The main states are:

\begin{itemize}
	\item \textbf{Untracked}: the file exists in the project directory but Git is unaware it exists, edits are not being watched/tracked. Files are usually only in this state when they are first created.
    \item \textbf{Unmodified}: Git knows about the file and is tracking the file for changes, but the file has not been edited since it was created or since the last commit.
    \item \textbf{Modified}: Git knows about the file and the file has been changed since it was created or since the last commit
    \item \textbf{Staged}: Git has been told to add the changes in this file to your next commit
    \item \textbf{Committed}: the file has been added to your last commit eg edits to it have been versioned/snapshotted. The file is now in the Unmodified state again.
\end{itemize}

\img{12cm}{resources/file-states.jpg}{1em}{The main states a file can be in according to Git}

\subsection{\texttt{.gitignore}}

The \texttt{.gitignore} file allows you to specify any files or directories you do not want tracked inside your project repo.
\\

On complex projects this is invaluable.
\\

It uses globbing patterns to match against file names. Examples are as follows:


\begin{minted}{bash}
    
    // By default, patterns match files in any directory
    Debug.log

    // The * is a wildcard that matches zero or more characters
    *.log

    // You can prepend a pattern with a double asterisk to match directories anywhere in the repository
    **.log

    //Prepending a slash matches files only in the repository root
    /debug.log

\end{minted}

\href{https://linux.die.net/man/7/glob}{Globbing patterns}
\href{https://www.atlassian.com/git/tutorials/saving-changes/gitignore}{Atlassian gitignore article}


\section{Basic commands}

\subsection{Creating a new repository}

Navigate to the directory you wish to get Git watching and type:

\begin{minted}{bash}
    git init
\end{minted}

This command creates a hidden local repository folder \texttt{.git}. This is where Git will store all its information about your project; which files there are, how they have changed over time, etc.
\\

Avoid having a Git repository within a Git repository. Eg you initialise Git in a parent directory and then also initialise Git in a subdirectory of the parent. Git will get in a total muddle and you'll get all sorts of funny side affects. 
\\

Anyway, after running \texttt{git init} successfully Git is ready to go, but is not yet tracking (version managing) anything.


\subsection{Finding out the state of a file}

The main command you will use to find the states of files in your repo, and what is going on in your repo generally:

\begin{minted}{bash}
    git status
\end{minted}

\img{12cm}{resources/git-status.jpg}{1em}{An example output from running git status}

You can also pass \texttt{git status} a flag, that gives a slightly different output.

\begin{minted}{bash}
    git status -s
\end{minted}

\img{12cm}{resources/git-status-s.jpg}{1em}{An example output from running git status -s}

The status codes are as follows:


\begin{itemize}
	\item ??  = untracked 	
    \item ' ' = unmodified (eg it's not showing up in the list)
    \item M   = modified
    \item A   = added
	\item D   = deleted
	\item R   = renamed
	\item C   = copied
	\item U   = updated but unmerged
\end{itemize}

VSCode also has some handy tools to help you track the status of your files too, and you will notice the same status flags on the right of each file name.
\\

\img{12cm}{resources/vscode-states.jpg}{1em}{The inbuilt Git tools in VSCode are really useful}


\subsection{Tracking our first file}

New files added to your project directory start off in an untracked state. They need to be tracked so Git watches them for changes.
\\

A file only needs to be tracked once and then Git will remember its existence forever - handy!
\\

Tracking a file and staging a file use the same command - more on staging files in a bit.

\begin{minted}{bash}
    git add {filename}
\end{minted}

\begin{infobox}{A few \texttt{git add} tricks}
    \begin{minted}{bash}
        // add multiple files in one command
        git add css/file1.css js/file2.js file3.html

        // add all files in a folder
        git add img/*

        // add all of type
        git add *.scss
    \end{minted}
\end{infobox}


\subsection{Committing our first file}

Now we've used \texttt{git add} to track a file and we've edited the file, we need to commit it.

\begin{minted}{bash}
    git commit -m "my first commit"
\end{minted}

Commits need to have a name and it is very important that you take the time to write good commit messages. Another person will need to read the message and fathom what changes to the code's behaviour are expected.
\\

\img{12cm}{resources/git_commit_messages.png}{1em}{Don't be like this person}

\subsection{Subsequent commits}

Most developers will regularly commit through their development time on a project.

For files that are already tracked (remember you only need to track them once when you first create them), you will need to ensure they are staged so Git knows to add them to your commit.
\\

Confusingly, staging a file uses the same command as tracking a file for the first time.
\\

\begin{minted}{bash}
    git add {filename}
\end{minted}

So you will find yourself using \texttt{git add} a lot! There are some shortcuts, so that you can stage and commit modified (and therefore tracked) files in one go:

\begin{infobox}{\texttt{git commit} arguments}
    \texttt{m} = Message associated with the changes in the commit
    \\
    \texttt{a} = All changed tracked files included in the commit
    \\

    Example usage, allowing the \texttt{git add} step to be skipped.

    \begin{minted}{bash}
        git commit -am "my second commit"
    \end{minted}
\end{infobox}

However, caution should be taken when you use this command. There will be times when you may not want to add all your modified files to a commit. You may have changed other files for testing or for other reasons.
\\

You must always be thoughtful of what files you are committing. If in doubt, take your time and run \texttt{git status} a whole ton of times to make sure you know what Git is doing (or about to do!).

\subsection{Changing a commit}

If you wish to amend the last commit you made, e.g. to change the commit message, then use:

\begin{minted}{bash}
    git commit --amend
\end{minted}


\subsection{Files with multiple statuses}

It is possible to edit a file, stage that change, then make a further change.
\\

The file will then be in both a staged state (first edit) and a modified state (further edits since first edits were staged).
\\

\texttt{git status -s} can help us decypher what has gone on here.

It's output:

\begin{minted}{bash}
 M README.md
MM index.html
A  css/styles.css
M  js/scripts.js
?? LICENSE.txt
\end{minted}

\texttt{⌃ } Left column: Staged changes column
\\
\texttt{ ⌃} Right column: Unstaged changes column
\\

\textbf{README.md}
\\
File has been edited but those modifications haven't been staged with \texttt{git add README.md} or commited with \texttt{git commit ...} yet.


\textbf{index.html}
\\
File has been edited, those modifications staged, and then has had further modifications made.
\\

\textbf{css/styles.css}
\\
File has been added into project (it's new) and a \texttt{git add css/styles.css} has been run to add the file into staging. The next \texttt{git commit ...} would add this file to the history.
\\

\textbf{js/scripts.js}
\\
File has been added edited and those modifications staged with \texttt{git add js/scripts.js}.
\\

\textbf{LICENSE.txt}
\\
Untracked file. Possibly new, but has never had \texttt{git add LICENSE.txt} run to start Git tracking it.
\\

\subsection{Seeing your commit history}

To see a list of your commits use:

\begin{minted}{bash}
    git log
\end{minted}

For a shorter list of recent commits use:

\begin{minted}{bash}
    git reflog
\end{minted}


\section{Additional resources}

\begin{itemize}[leftmargin=*]
    \item \href{https://github.com/alphagov/styleguides/blob/master/git.md}{GDS Git style guide}
    \item \href{https://chris.beams.io/posts/git-commit/}{How to write a good Git commit message}
	\item \href{https://thoughtbot.com/blog/5-useful-tips-for-a-better-commit-message}{5 tips for a better commit message}
	\item \href{https://linux.die.net/man/7/glob}{Globbing patterns}
	\item \href{https://www.atlassian.com/git/tutorials/saving-changes/gitignore}{Atlassian gitignore article}
\end{itemize}

