\section{Commands}

\subsection{Starting a New Project}

Git won't operate on your project unless you initialise it:

\begin{minted}{bash}
    git init
\end{minted}

At this point Git is ready to go, but is not yet tracking (version managing) anything.
\\

What this command does do is create a hidden local repository folder \texttt{.git}
\\

This is where Git will store all its information about your project; which files there are, how they have changed over time, etc.

\subsection{Tracking Our First File}

New files added created the project start off in an untracked state, and need to be staged and committed before Git will track their state in future.
\\

\begin{minted}{bash}
    git add {filename}
\end{minted}

\begin{infobox}{A few \texttt{git add} tricks}
    \begin{minted}{bash}
        // add multiple files in one command
        git add css/file1.css js/file2.js file3.html

        // add all files in a folder
        git add img/*

        // add all of type
        git add *.scss
    \end{minted}
\end{infobox}

\subsection{Committing Our First File}

Now we've used \texttt{git add} to track a file, we need to commit it.

\begin{minted}{bash}
    git commit -m "my first commit"
\end{minted}

\begin{infobox}{\texttt{git commit} arguments}
    \texttt{m} = Message associated with the changes in the commit
    \\
    \texttt{a} = All changed tracked files included in the commit
    \\

    Example usage, allowing the \texttt{git add} step to be skipped.

    \begin{minted}{bash}
        git commit -am "my second commit"
    \end{minted}
\end{infobox}

\subsection{Changing a Commit}

If you wish to amend the last commit you made, e.g. to change the commit message, then use:

\begin{minted}{bash}
    git commit --amend
\end{minted}

\section{File Lifecycle \& Staging}

Once files are tracked by Git (have been \texttt{git add}ed) then they go through these stages.

\begin{itemize}
    \item \textbf{Unmodified}: file has no changes from last commit
    \item \textbf{Modified}: \texttt{git status} reports file has been edited
    \item \textbf{Staged}: \texttt{git add \{filename\}} has been run to stage the change, ready to go into next commit
    \item \textbf{Committed/unchanged}: \texttt{git commit ...} creates a new snapshot of those edits, and file goes back to being unchanged (from the last commit we just made)
\end{itemize}

\img{12cm}{resources/untracked.png}{1em}{Life cycle of tracked files in a Git version managed project}

\subsection{Files with Multiple Statuses}

It is possible to edit a file, stage that change, then make a further change.
\\

The file will then be in both a staged state (first edit) and a modified state (further edits since first edits were staged).
\\

\texttt{git status -s} can help us decypher what has gone on here.

It's output:

\begin{minted}{bash}
 M README.md
MM index.html
A  css/styles.css
M  js/scripts.js
?? LICENSE.txt
\end{minted}

\texttt{⌃ } Left column: Staged changes column
\\
\texttt{ ⌃} Right column: Unstaged changes column
\\

\textbf{README.md}
\\
File has been edited but those modifications haven't been staged with \texttt{git add README.md} or commited with \texttt{git commit ...} yet.


\textbf{index.html}
\\
File has been edited, those modifications staged, and then has had further modifications made.
\\

\textbf{css/styles.css}
\\
File has been added into project (it's new) and a \texttt{git add css/styles.css} has been run to add the file into staging. The next \texttt{git commit ...} would add this file to the history.
\\

\textbf{js/scripts.js}
\\
File has been added edited and those modifications staged with \texttt{git add js/scripts.js}.
\\

\textbf{LICENSE.txt}
\\
Untracked file. Possibly new, but has never had \texttt{git add LICENSE.txt} run to start Git tracking it.
\\
